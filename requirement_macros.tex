% Command to generate a new subsection for a level of requirements.
% Arguments:
%   - Title: Name of the level (e.g. System-High-Level)
%   - Group-ID: ID of the level (e.g. Sys-HL)
\newcommand{\group}[2]{
    \newpage
    \subsection{#1}
    \renewcommand{\domain}{#2}
}

% Set the domain to an initial value so that \group can redefine it.
\newcommand{\domain}{NO DOMAIN SET}

% Command to generate a new subsubsection for a requirement.
% The name is of the form [group]_REQ_[num], where group
% can be set via a call to \group.
% Arguments:
%   - ID: usually a number, should be unique for this group
%   - Title: The short description of the requirement
%   - Description: The full description of the requirement
%   - Parent requirements: Links to the parent requirements
%   - Partition: Partitioning of this requirement, ignored if empty
\newcommand{\req}[5]{
    \subsubsection*{Requirement \domain-REQ-\textbf{#1}: #2} \label{\domain-REQ-#1}
    \paragraph{Description:}
    #3

    \paragraph{Parent Requirements:}
    #4

    \ifthenelse{\equal{#5}{}}{}{
        \paragraph{Partition:}
        #5
    }
}

% Command to link to a parent requirement
% Arguments:
%   - ID: ID of the parent requirement.
\newcommand{\parent}[1]{\hyperref[#1]{#1}}

% Command to generate a new subsubsection for a design decision.
% The name is of the form [group]_DD_[num], where group
% can be set via a call to \group.
% Arguments:
%   - ID: usually a number, should be unique for this group
%   - Title: The short description of the requirement
%   - Description: The full description of the requirement
%   - Reasoning: Reason for the choice of design decision.
\newcommand{\dd}[4]{
    \subsubsection*{Design Decision \domain-DD-\textbf{#1}: #2} \label{\domain-DD-#1}
    \paragraph{Description:}
    #3

    \paragraph{Reasoning:}
    #4
}
