% Command to generate a new subsubsection for a requirement.
% The name is of the form [group]_REQ_[num], where group
% can be set via a call to \group.
% Arguments:
%   - ID: usually a number, should be unique for this group
%   - Title: The short description of the requirement
%   - Description: The full description of the requirement
%   - Parent requirements: Links to the parent requirements
%   - Partition: Partitioning of this requirement, ignored if empty
\newcommand{\req}[5]{
    \fcolorbox{teal}{teal!30}{
        \begin{minipage}{\textwidth}
            \subsection*{#1: #2} \label{#1}
            \paragraph{Description:}
            #3

            \paragraph{Parent Requirements:}
            #4

            \ifthenelse{\equal{#5}{}}{}{
                \paragraph{Partition:}
                #5
            }
        \end{minipage}
    }

    \vspace{.5cm}
    
}

% Command to link to a parent requirement
% Arguments:
%   - ID: ID of the parent requirement.
\newcommand{\parent}[1]{\hyperref[#1]{#1}}

% Command to generate a new subsubsection for a design decision.
% The name is of the form [group]_DD_[num], where group
% can be set via a call to \group.
% Arguments:
%   - ID: usually a number, should be unique for this group
%   - Title: The short description of the requirement
%   - Description: The full description of the requirement
%   - Reasoning: Reason for the choice of design decision.
\newcommand{\dd}[4]{
    \fcolorbox{teal}{teal!30}{
        \begin{minipage}[l]{\textwidth}
            \subsection*{#1: #2} \label{#1}
            \paragraph{Description:}
            #3

            \paragraph{Reasoning:}
            #4
        \end{minipage}
    }

    \vspace{.5cm}
    
}
